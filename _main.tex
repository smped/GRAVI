% Options for packages loaded elsewhere
\PassOptionsToPackage{unicode}{hyperref}
\PassOptionsToPackage{hyphens}{url}
%
\documentclass[
]{book}
\usepackage{amsmath,amssymb}
\usepackage{iftex}
\ifPDFTeX
  \usepackage[T1]{fontenc}
  \usepackage[utf8]{inputenc}
  \usepackage{textcomp} % provide euro and other symbols
\else % if luatex or xetex
  \usepackage{unicode-math} % this also loads fontspec
  \defaultfontfeatures{Scale=MatchLowercase}
  \defaultfontfeatures[\rmfamily]{Ligatures=TeX,Scale=1}
\fi
\usepackage{lmodern}
\ifPDFTeX\else
  % xetex/luatex font selection
\fi
% Use upquote if available, for straight quotes in verbatim environments
\IfFileExists{upquote.sty}{\usepackage{upquote}}{}
\IfFileExists{microtype.sty}{% use microtype if available
  \usepackage[]{microtype}
  \UseMicrotypeSet[protrusion]{basicmath} % disable protrusion for tt fonts
}{}
\makeatletter
\@ifundefined{KOMAClassName}{% if non-KOMA class
  \IfFileExists{parskip.sty}{%
    \usepackage{parskip}
  }{% else
    \setlength{\parindent}{0pt}
    \setlength{\parskip}{6pt plus 2pt minus 1pt}}
}{% if KOMA class
  \KOMAoptions{parskip=half}}
\makeatother
\usepackage{xcolor}
\usepackage{longtable,booktabs,array}
\usepackage{calc} % for calculating minipage widths
% Correct order of tables after \paragraph or \subparagraph
\usepackage{etoolbox}
\makeatletter
\patchcmd\longtable{\par}{\if@noskipsec\mbox{}\fi\par}{}{}
\makeatother
% Allow footnotes in longtable head/foot
\IfFileExists{footnotehyper.sty}{\usepackage{footnotehyper}}{\usepackage{footnote}}
\makesavenoteenv{longtable}
\usepackage{graphicx}
\makeatletter
\def\maxwidth{\ifdim\Gin@nat@width>\linewidth\linewidth\else\Gin@nat@width\fi}
\def\maxheight{\ifdim\Gin@nat@height>\textheight\textheight\else\Gin@nat@height\fi}
\makeatother
% Scale images if necessary, so that they will not overflow the page
% margins by default, and it is still possible to overwrite the defaults
% using explicit options in \includegraphics[width, height, ...]{}
\setkeys{Gin}{width=\maxwidth,height=\maxheight,keepaspectratio}
% Set default figure placement to htbp
\makeatletter
\def\fps@figure{htbp}
\makeatother
\setlength{\emergencystretch}{3em} % prevent overfull lines
\providecommand{\tightlist}{%
  \setlength{\itemsep}{0pt}\setlength{\parskip}{0pt}}
\setcounter{secnumdepth}{5}
\usepackage{booktabs}
\ifLuaTeX
  \usepackage{selnolig}  % disable illegal ligatures
\fi
\usepackage[]{natbib}
\bibliographystyle{plainnat}
\IfFileExists{bookmark.sty}{\usepackage{bookmark}}{\usepackage{hyperref}}
\IfFileExists{xurl.sty}{\usepackage{xurl}}{} % add URL line breaks if available
\urlstyle{same}
\hypersetup{
  pdftitle={GRAVI User Guide (v0.1.6)},
  pdfauthor={Stephen Pederson},
  hidelinks,
  pdfcreator={LaTeX via pandoc}}

\title{GRAVI User Guide (v0.1.6)}
\author{Stephen Pederson}
\date{2023-07-09}

\begin{document}
\maketitle

{
\setcounter{tocdepth}{1}
\tableofcontents
}
\hypertarget{introduction}{%
\chapter{Introduction}\label{introduction}}

This book is the primary documentation for running the GRAVI (Gene Regulatory Analysis using Variable Inputs) workflow.
This workflow is managed using \texttt{snakemake} (\textgreater v7.7) for running locally or on any HPC, and is designed to minimally take one ChIP target under at least two conditions.
There is no theoretical upper limit to the number of ChIP targets which can be analysed, although the practicalities of interpretation will dictate this.

\textbf{Required} additional inputs are:

\begin{itemize}
\tightlist
\item
  A gtf with gene, transcript and exon-level annotations. The workflow has been tested extensively with \href{https://www.gencodegenes.org/}{Gencode annotations} and these are preferred.
\item
  A blacklist should also be provided as a bed file. These can easily be obtained from \href{https://github.com/Boyle-Lab/Blacklist/tree/master/lists}{here}
\end{itemize}

In addition to \(\geq 1\) ChIP targets, \textbf{optional} input includes:

\begin{itemize}
\tightlist
\item
  Differential Expression results from a single RNA-Seq experiment as a \texttt{tsv/csv}
\item
  Any type of genomic feature derived or obtained externally (\texttt{*gtf})
\item
  HiC Interactions (\texttt{*.bedpe})
\item
  Additional coverage tracks for visualisation, such as those produced by key histone marks (\texttt{*bigwig})
\end{itemize}

The GRAVI workflow itself will

\begin{itemize}
\tightlist
\item
  Annotate the genome using a custom, transcript-focussed approach
\item
  Identify peaks using \texttt{macs2\ callpeak}
\item
  Perform differential binding analysis for each ChIP target \& requested comparisons
\item
  Compare differential binding results across ChIP targets or samples (Pair-wise comparisons)
\item
  Perform enrichment analyses at all steps of the workflow
\end{itemize}

The primary output is a series of \texttt{html} pages generated from \texttt{rmarkdown} files, along with key figures and tables able to be shared amongst collaborators and incorporated directly into publications.

\hypertarget{workflow}{%
\chapter{Workflow Description}\label{workflow}}

The GRAVI workflow performs multiple steps, most of which depend on those conducted previously.
The workflow management software \texttt{snakemake}\citep{snakemake} is used to run the complete workflow, giving the capacity to be run on local servers or HPC systems.

The \texttt{snakemake} DAG (directed acyclic graph) of the workflow is always included in the compiled document, however a simplified version is presented below.

\label{fig:unnamed-chunk-3}Simplified DAG of the GRAVI workflow. Key steps are numbered and shown in black. Inputs are shown in blue, whilst key outputs are shown in coral.

The workflow produces a series of HTML reports as a larger webpage, inspired by the excellent \texttt{workflowr}\citep{workflowr} package but instead relying directly on \texttt{rmarkdown::render\_site()}\citep{R-rmarkdown}.
The compiled html pages will be placed in the \texttt{docs} directory, with all source \texttt{rmarkdown} files being placed in the \texttt{analysis} directory.
Key additional outputs (e.g.~bed/csv files) are placed in the \texttt{output} directory.
The standardised directory layout is shown in Section \ref{directories}.
The complete R environment from every compiled HTML page is also saved in \texttt{output/envs}, however these are marked as temporary files by the workflow and will be deleted when removing \texttt{temp} output (\texttt{snakemake\ -\/-delete-temp-output\ -\/-cores\ 1})

\hypertarget{annotation}{%
\section{Annotation Setup}\label{annotation}}

\hypertarget{genomic-regions}{%
\subsection*{Genomic Regions}\label{genomic-regions}}

Defining genomic regions is a key part of the setup for analysis.
Under the GRAVI workflow, a series of \emph{non-overlapping genomic regions} are defined which characterise the most likely role/aspect for each specific region.
As these will be unique to each Gencode build, and also include any optional RNA-Seq data, this step is performed for every experiment.
This step follows the default process as defined in the function \texttt{defineRegions()} from the Bioconductor package \texttt{extraChIPs}

\begin{enumerate}
\def\labelenumi{\arabic{enumi}.}
\tightlist
\item
  \emph{Promoters}: By default these are defined as -1500/+500bp from every distinct TSS.
  If promoters from two or more transcripts overlap, they are merged into a single promoter.
  These ranges can be easily changed to increase or decrease the size using the YAML as described in Section \ref{params-yml}
\item
  \emph{Upstream Promoters}: These are extended promoter regions up to 5kb upstream by default, which again can be changed to suit
\item
  \emph{Exons} are defined as any exon not overlapping a promoter or upstream promoter
\item
  \emph{Introns} are defined as any transcribed sequence not overlapping a promoter, upstream promoter or exon
\item
  \emph{Intergenic Regions}: are divided into two subsets, neither of which are permitted to overlap any previously defined regions

  \begin{itemize}
  \tightlist
  \item
    Within 10kb of a gene. Again this distance is customisable
  \item
    Beyond 10kb of a gene.
  \end{itemize}
\end{enumerate}

During the characterisation of all the above regions, mappings to associated genes and transcripts is retained and included in the subsequent \texttt{GenomicRanges} object.

\hypertarget{other-steps}{%
\subsection*{Other Steps}\label{other-steps}}

Additional steps performed during preparation of the annotations are to ensure that all external features, treatment groups and defined genomic regions have colours assigned, which will then propagate through the workflow for consistency of visualisation.
External features and/or HiC data is summarised if provided and the association between these datasets and defined genomic regions is also provided as part of the output.

Greylisted regions are also determined and prepared for exclusion through the workflow, along with the provided blacklisted regions.

\hypertarget{peak-calling}{%
\section{Peak Calling}\label{peak-calling}}

This section of the workflow uses \texttt{macs2\ callpeak}\citep{macs2} with default parameters.
Peaks are called on each individual sample and by merging samples within treatment groups for each provided ChIP target.
One summary report will be produced each ChIP target specified in the \texttt{target} column of \texttt{samples.tsv}, which will assess all treatments within a ChIP target.

\hypertarget{quality-assessment}{%
\subsection*{Quality Assessment}\label{quality-assessment}}

Basic QC statistics such as Library Size, Read Lengths, Total Detected Peaks and Fraction Of Reads In Peaks (FRIP) are provided in tabular and visual form.
Plots showing sample-specific GC content and Cross Correlations are also provided to enable visual identification of any outlier samples which can be excluded manually, or handled in any other suitable manner.

One feature of the GRAVI workflow is to compare between replicates within a treatment group to determine any low-quality samples.
Taking the median number of peaks within a sample group (\(np_i\)), and sample with \(>N\)-fold or \(<N\)-fold \(np_i\) peaks will be marked as an outlier.
By default, this is set to 10-fold using the parameter \texttt{outlier\_threshold} (Section \ref{config-yml}) and can be disabled by setting this parameter to \texttt{.inf}, which will be passed to \texttt{R} as the value \(\infty\).
Given that some ChIP targets may be cytoplasmic and yield zero peaks in some treatment groups, samples with zero peaks can also be allowed by setting \texttt{allow\_zero} to be \texttt{true}.

\hypertarget{results}{%
\subsection*{Results}\label{results}}

\textbf{Individual Replicates}: UpSet plots\citep{upsetr} of all samples are first produced to enable identification of any possible mislabelling.
Venn Diagrams are then produced for all replicates within each treatment groups (if \(n \leq3\)), or UpSet plots are produced for \(n \geq4\) replicates to show consistency between replicates.

\textbf{Treatment Peaks}: During this step of the workflow a set of \emph{treatment-specific peaks} will be produced for each ChIP target.
The set of peaks obtained by merging samples will be compared to all individual samples, and only those peaks identified in at least \(100*p\%\) of the samples which passed QC will be included in the set of Treatment peaks.
This parameter can be set in \texttt{config.yml} as the parameter \texttt{min\_prop\_reps} (Section \ref{config-yml}) with the default of 0.5 indicating that peaks must be detected in 1 or more samples (if N = 2), two or more samples (if N = 3, 4) etc.

\textbf{Union Peaks}: After defining the treatment-specific peaks, these are combined to define the set of \emph{union peaks}.
Treatment peaks are merged across conditions in an inclusive manner, such that the range of each union peak encompasses the entire region covered by all overlapping treatment peaks.
Union peaks are then used as the `universe' of peaks against which treatment-specific behaviours are compared.

Peaks overlapping a blacklisted or greylisted region are excluded at all steps of the analysis.

The final HTML report produced for each target will contain:

\begin{enumerate}
\def\labelenumi{\arabic{enumi}.}
\tightlist
\item
  A Venn Diagram showing the overlap in Union peaks and whether one or more Treatment peaks overlaps the Union peak. In the case of 4 or more treatments, an UpSet plot will be produced instead
\item
  Distance to TSS plots as actual and cumulative distributions using union peaks.
\item
  A pie chart describing the distribution of union peaks within the genomic regions defined previously
\end{enumerate}

Additional plots will be produced in the presence of external data

\begin{enumerate}
\def\labelenumi{\arabic{enumi}.}
\tightlist
\item
  The distribution of union peaks within external features will be shown \emph{if external features are provided}
\item
  The distribution of union peaks within external features and genomic regions will be shown \emph{if external features are provided}
\end{enumerate}

Highly-ranked union and treatment-specific Treatment peaks are also shown in their genomic context alongside regulatory/external features and nearby gene models.

\hypertarget{key-outputs}{%
\subsection*{Key Outputs}\label{key-outputs}}

\begin{itemize}
\tightlist
\item
  Union peaks will be exported as \texttt{output/macs2/\textless{}target\textgreater{}/\textless{}target\textgreater{}\_union\_peaks.bed}
\item
  Treatment peaks will be exported as individual bed files for each treatment group (\texttt{output/macs2/\textless{}Ctarget\textgreater{}/\textless{}target\textgreater{}\_\textless{}treat\textgreater{}\_treatment\_peaks.bed}), as well as combined in a single \texttt{RDS} object for faster parsing into an \texttt{R} environment as \texttt{output/macs2/\textless{}target\textgreater{}/\textless{}target\textgreater{}\_treatment\_peaks.rds}
\item
  BigWig files will be generated from the bedgraph files output by \texttt{mcas2\ callpeak} using the \texttt{-\/-SPMR} tag, for each individual sample and for each set of peaks called by merging samples from the same group
\item
  Additional BigWig files will be produced from the bedgraph files output by \texttt{macs2\ bdgcmp}, representing fold-enrichment over the input sample for each set of grouped samples
\end{itemize}

\hypertarget{differential-signal}{%
\section{Differential Signal}\label{differential-signal}}

This step provides much of the uniqueness to the GRAVI workflow combining approaches from \texttt{macs2}\citep{macs2}, \texttt{qsmooth}\citep{qsmooth}, \texttt{csaw}\citep{csaw}, \texttt{limma}\citep{limma}, \texttt{edgeR}\citep{edger} and \texttt{ihw}\citep{ihw} whilst relying heavily on the infrastructure provided by \href{https://bioconductor.org/packages/release/bioc/html/extraChIPs.html}{\texttt{extraChIPs}}.
A sliding window approach as advocated by Lun \& Smyth (2014) is the primary strategy used for detection of differetial signal.

All two-factor comparisons of interest can specified via the YAML file (Section \ref{config-yml}) with more complicated layouts specified as below

\begin{verbatim}
contrasts:
  - ["control", "treat1"]
  - ["control", "treat2"]
\end{verbatim}

Differential signal analysis will be performed for every ChIP target where both treatment groups are found from each specified contrast.
Two possible approaches are available, with these being 1) \texttt{sq-lt} (Smooth-Quantile limma-trend) or 2) \texttt{ls-ql} (Library-Size Quasi-Likelihood).

\label{fig:unnamed-chunk-4}Overview of steps in the differential signal workflow. Primary R packages used for each step are indicated in brackets. Integration with differential expression results (RNA-Seq) is an option step only performed if RNA-Seq data is provided.

\hypertarget{sliding-windows}{%
\subsection*{Sliding Windows}\label{sliding-windows}}

By default, sliding windows will be defined based on the estimated fragment length so that the window size is slightly wider than the fragment length (\(w_\text{size}\)), and the step size is \(w_\text{step} = w_\text{size} / 3\).
This usually leads to window sizes which are multiples of 30nt, e.g.~90, 120, 150, 180, 210, 240 etc.

Alignments are initially counted across autosomes and sex chromosomes, explicitly excluding scaffolds and mitochondrial alignments, as well as regions contained in black/greylists.
Windows are discarded if the total number of alignments across all samples are \(<n\), where \(n\) represents the total number of samples in the current analysis.
Windows which are more likely to contain noise than true signal are then discarded using \texttt{extraChIPs::dualFilter()} in combination with the \emph{Union peaks} from Section \ref{peak-calling}.
The set of Union peaks is used as a guide for setting the inclusion/exclusion thresholds which are based on 1) Overall signal intensity and 2) Enrichment over input.
Thresholds for each measure are defined such that the proportion \(q\) of windows which overlap a Union peak are returned, with windows which pass \emph{both} inclusion thresholds are returned.
This parameter itself is set in \texttt{config.yml} as \texttt{filter\_q} as described in Section \ref{config-yml}.
In general, values in the range \(0.4 < q < 0.6\) perform well as the sliding windows around the margins of the Union peaks will be discarded at this point.
Higher values will lead to large numbers of windows being retained which overlap peak margins and are relatively uninformative.

\hypertarget{normalisation}{%
\subsection*{Normalisation}\label{normalisation}}

Under the \texttt{sq-lt} method, logCPM values are first calculated using the \emph{total library size} for each sample.
These values are then normalised using Smooth Quantile Normalisation\citep{qsmooth} specifiying treatment groups as the factor of interest and allowing for normalisation within and between groups.
This can be particularly useful in the scenario where a transcription factor (TF) is primarily cytoplasmic in one treatment group, before shifting to the nucleus in response to treatment.
Plots showing qsmooth weights, pre/post logCPM, pre/post RLE \citep{rle} and pre/post PCA are produced in this section of the workflow.

Alternatively, under the \texttt{ls-ql} methodology, only the \emph{total library size} is provided to the GLM fitting stage with no further normalisation strategies being employed.
This is analogous to the default \texttt{edgeR} option provided using DiffBind \citep{diffbind}.

\hypertarget{hypothesis-testing}{%
\subsection*{Hypothesis Testing}\label{hypothesis-testing}}

Given the normalised logCPM values being analysed under the \texttt{sq-lt} approach, an appropriate strategy for detection of differential signal is the \texttt{limma-trend} method \citep{voom} as established for RNA-Seq analysis.
Under the \texttt{ls-ql} approach, raw counts are analysed using the Quasi-Likelihood approach for Negative Binomial data \citep{qlf}, again as developed for RNA-Seq data, and as offered when using DiffBind \citep{diffbind}.

Both strategies are performed on \emph{all retained windows} using a \textbf{range-based Null Hypothesis} \citep{treat}, such that a minimum change can be specified, below which changes in signal are not considered to be of interest.
Under this statistical approach, the Null Hypothesis would be

\[
H_0: -\lambda \leq \mu \leq \lambda
\]

with the alternate hypothesis being

\[
H_A: |\mu| > \lambda
\]

By default, the value \(\lambda = \log_2(1.2)\) is used, which denotes a 20\% change in signal as the point where sites become of interest.
This value is set in \texttt{config.yml} as the parameter \texttt{fc} (Default: \texttt{fc\ =\ 1.2}), and setting \texttt{fc\ =\ 1} would return the statistical approach to be a point-based Null Hypothesis \(H_0: \mu = 0\).

After performing this statistical test, overlapping windows are \emph{merged} taking the individual window with the \textbf{highest signal} as the representative window for the merged region.
As this value is independent of the statistical test\citep{csaw}, the resultant set of p-values are FDR-adjusted using the Benajmini-Hochberg approach\citep{fdr}, giving a set of FDR-adjusted p-values for all merged regions.

\hypertarget{independent-hypothesis-weighting}{%
\subsection*{Independent Hypothesis Weighting}\label{independent-hypothesis-weighting}}

The Independent Hypothesis Weighting approach\citep{ihw} suggests the a set of p-values can be partitioned by any independent variable, with weights assigned to each partition, and these weighted p-values can then be adjusted using conventional strategies, such as the FDR.
Under the GRAVI workflow, four possibilities for partitioning the p-values obtained after merging regions are provided, and these can be specified in the \texttt{ihw} parameter of \texttt{config.yml}.
The options are

\begin{enumerate}
\def\labelenumi{\arabic{enumi}.}
\tightlist
\item
  \texttt{ihw:\ "targets"} where union peaks from all other ChIP targets included in the larger GRAVI workflow are used to partition p-values. As union peaks are treatment-agnostic, these simply provide a scaffold defining the presence/absence of all other ChIP targets under any treatment condition, in combination across all ChIP targets.
\item
  \texttt{ihw:\ "regions"} where previously defined genomic regions are used to partition the p-values
\item
  \texttt{ihw:\ "features"} where any external features supplied are used for the partitioning
\item
  \texttt{ihw:\ "none"} where no partitioning is performed and the standard FDR-adjusted p-values are used to determine differential signal status
\end{enumerate}

The default approach implemented by the IHW authors suggests that p-value partitions must be \textgreater{} 1000, and as such, all the above approaches collapse smaller groups until the smallest group contains at least 1000 p-values (i.e.~merged regions).
In the case of ChIP targets which do not bind in a promiscuous manner, the IHW step may make minimal difference to the results.

\hypertarget{assigning-genes-to-windows}{%
\subsection*{Assigning Genes To Windows}\label{assigning-genes-to-windows}}

After merging of neighbouring sliding windows, genes are assigned to each merged region.
This is performed using annotated genomic regions, any external features and HiC data, via the function \texttt{extraChIPs::mapByFeatures()}.

Under this approach:

\begin{enumerate}
\def\labelenumi{\arabic{enumi}.}
\tightlist
\item
  Regions which overlap a promoter are assigned to the \emph{genes associated directly with that promoter}
\item
  Regions which overlap an enhancer are assigned to \emph{all genes within 100kb}
\item
  Regions which overlap any HiC interactions are assigned to \emph{all genes connected by the interactions}
\item
  Regions with no assignment from steps 1-3 are assigned to \emph{all directly overlapping genes}, or the \emph{nearest gene within 100kb}
\end{enumerate}

If no HiC data is included, step 3 is not performed

\hypertarget{presentation-of-key-results}{%
\subsection*{Presentation of Key Results}\label{presentation-of-key-results}}

The above steps essentially complete the detection of differential signal regions and assignment of these to regulatory target genes.
A series of visualisations are then provided including:

\begin{enumerate}
\def\labelenumi{\arabic{enumi}.}
\tightlist
\item
  MA \& Volcano plots
\item
  Profile Heatmaps for sites with Increased/Decreased target signal
\item
  Results summarised by chromosome
\item
  Signal/logFC/differential signal partitioned by genomic region
\end{enumerate}

If \emph{external features are provided}, the plots from step 4 are replicated using external features to partition results.
A combined summary of the results by genomic region and external features is also included.

A series of summary tables, along with the 200 most highly ranked regions are included.
Results of differential signal windows and the genes assigned to them are also exported as a CSV for sharing with collaborators.
Genomic plots for unique regions based on signal strength (logCPM), changed signal (logFC) and statistical support (FDR) are also provided as part of the default output.

\hypertarget{enrichment-testing}{%
\subsection*{Enrichment Testing}\label{enrichment-testing}}

Using the gene-sets and pathways specified in \texttt{params.yml} four enrichment analyses are performed using \texttt{goseq}\citep{goseq}.
These are

\begin{enumerate}
\def\labelenumi{\arabic{enumi}.}
\tightlist
\item
  Comparison of genes mapped to a site with \textbf{target signal} and genes which are not mapped to a region with target signal
\item
  Comparison of genes mapped to a \textbf{differential signal site} against genes mapped to a site with detected signal, but which is considered unchanged
\item
  Comparison of genes mapped to a site with \textbf{increased signal} against genes mapped to a site, but with no differential signal
\item
  Comparison of genes mapped to a site with \textbf{decreased signal} against genes mapped to a site, but with no differential signal
\end{enumerate}

Gene-width is used as an offset for biased sampling in the analysis of bound target only, as longer genes are more likely to have a peak mapped to them.
For all other analyses, no offset term is included and the standard Hypergeometric test is performed.

All results are presented as searchable, interactive HTML tables including which genes from each pathway are mapped to the relevant set of regions from which signal was detected.
Where enough pathways are detected as significantly enriched, network plots are also produced deriving distances between nodes based on shared, differential-signal genes using the \emph{overlap coefficient.}

\hypertarget{rna-seq-data}{%
\subsection*{RNA-Seq Data}\label{rna-seq-data}}

If results from a differential expression (DE) analysis are included (which can also be microarray results), a series of additional analyses are performed.
Firstly, the relationship between detected genes and the genomic regions where target signal has been detected are characterised.
If external features are also provided, this is repeated incorporating these features.

P-values from DE analysis are partitioned by differential signal status.
No further IHW is performed, but this can still provide a clear visual clue as to the relationship between target signal and differential gene expression.

Genomic regions for the 5 genes most highly ranked for differential expression, and with \(>1\) target-bound window assigned to them are plotted.

Given the unpredictability of an RNA-Seq dataset and the number of genes considered to be differentially expressed, bound and differential signal windows are used as gene sets to perform GSEA\citep{fgsea} using sites directly, and sites partitioned by genomic region and external feature (if provided).
GSEA is performed 1) taking direction of differential expression into account, 2) ranking genes purely by significance (i.e.~without direction).

Finally, standalone GSEA results from the RNA-seq dataset (incorporating direction of change and all requested \texttt{MSigDB} pathways) are compared to enrichment (i.e.~\texttt{goseq}) results for differential signal regions.
As these are essentially orthogonal viewpoints on gene regulation, P-values from each analysis are combined using Wilkinson's \texttt{maximump()} method and the resultant set of p-values adjusted using the specified method and requested threshold for \(\alpha\)
Any common pathways are given in an interactive HTML table along with network plots based on shared regulatory targets also found in the leading edge fro GSEA.

\hypertarget{key-outputs-1}{%
\subsection*{Key Outputs}\label{key-outputs-1}}

Key output files produced by this step of the workflow are:

\begin{itemize}
\tightlist
\item
  Differential signal results (by window): \texttt{output/differential\_binding/\textless{}target\textgreater{}/\textless{}target\textgreater{}\_\textless{}control\textgreater{}\_\textless{}treat\textgreater{}\_differential\_binding.rds}
\item
  Differential signal results (by gene): \texttt{output/differential\_binding/\textless{}target\textgreater{}/\textless{}target\textgreater{}\_\textless{}control\textgreater{}\_\textless{}treat\textgreater{}\_differential\_binding.csv.gz}
\item
  Regions with increased target signal: \texttt{output/differential\_binding/\textless{}target\textgreater{}/\textless{}target\textgreater{}\_\textless{}control\textgreater{}\_\textless{}treat\textgreater{}\_up.bed}
\item
  Regions with decreased target signal: \texttt{output/differential\_binding/\textless{}target\textgreater{}/\textless{}target\textgreater{}\_\textless{}control\textgreater{}\_\textless{}treat\textgreater{}\_down.bed}
\end{itemize}

\hypertarget{pairwise-comparisons}{%
\section{Pairwise Comparisons}\label{pairwise-comparisons}}

By default, pair-wise comparisons are performed between all differential signal results.
If ChIP target TF1 has samples in Control and Treat1, whilst ChIP target TF2 has samples in Control, Treat1 and Treat2, with comparisons being requested for Treat1 vs.~Control and Treat2 Vs Control, the following pair-wise comparisons will be automatically performed:

\begin{enumerate}
\def\labelenumi{\arabic{enumi}.}
\tightlist
\item
  TF1 (Treat1 Vs Control) and TF2 (Treat1 Vs Control)
\item
  TF1 (Treat1 Vs Control) and TF2 (Treat2 Vs Control)
\item
  TF2 (Treat1 Vs Control) and TF2 (Treat2 Vs Control)
\end{enumerate}

Given the automated nature of the workflow, this represents the simplest approach and any redundant comparisons are simply able to be ignored by the user.

\hypertarget{comparison-of-peaks}{%
\subsection*{Comparison of Peaks}\label{comparison-of-peaks}}

The pair-wise comparisons module initially compares union peaks between the two targets as a Venn Diagram, with the sets of (up to) four treatment-specific peaks being compared using an UpSet plot.

\hypertarget{comparison-of-differentially-bound-windows}{%
\subsection*{Comparison of Differentially Bound Windows}\label{comparison-of-differentially-bound-windows}}

Pair-wise comparison of two ChIP targets requires more nuance than simply looking for sites where both are changed.
A universal set of windows is first obtained across all windows retained in both targets.
These are then classified as either \emph{Up}, \emph{Down}, \emph{Unchanged} or \emph{Undetected} for each ChIP target.
Using both targets, the universal windows are then classified based on \textbf{both targets}.
This is particularly important given the hypothesised role of ChIP targets which can act as pioneer factors\citep{pioneerfactors}, or those which bind in complex and sequester other factors\citep{hickeynatmed}.

The classification of each window is initially based on significant FDR-adjusted p-value using the range-based H\textsubscript{0} in at least one comparison.
In order to ensure more accurate assignment of windows in the secondary comparison, the FDR-adjusted p-values using a \emph{point-based} H\textsubscript{0} are used, in conjunction with an estimated logFC beyond the range \(\pm\lambda\), i.e.~\(|\widehat{\text{logFC}}| > \lambda\).
This reduces the number of regions incorrectly classified as unchanged in one comparison due to the use of the range-based H\textsubscript{0}, as is common for all situations where threshold are applied.

The combined behaviours of both ChIP targets is then compared directly, described by genomic region and external features (if provided).
Distances between the windows with representative statistics (i.e.~maximal signal) are determined where both targets are present.
The combined changes in signal are then compared as a complete set for all windows where both targets are detected, as well as broken down by genomic region and external features.
Distances between the windows where the maximal signal was observed are also presented

\hypertarget{combined-visualisations}{%
\subsection*{Combined Visualisations}\label{combined-visualisations}}

Profile heatmaps for all behavioural groups which are detected (e.g.~TF1 Up - TF2 Up, TF1 Up - TF2 - Unchanged etc) are all presented.

The genomic windows for which the two factors are present are visualised based on the combined strongest signal, and the combined largest change.
Using only combinations of Up/Down/Unchanged, six windows for each are presented.
As with all genomic visualisations, any coverage provided as an external track (e.g H3K27ac or ATAC-seq signal) will be added to all plots.

\hypertarget{enrichment-analysis}{%
\subsection*{Enrichment Analysis}\label{enrichment-analysis}}

The same gene-sets from MSigDB\citep{msigdb} as used previously are then used for enrichment testing, using genes as mapped to windows during previous steps.
Enrichment testing again uses \texttt{goseq} with gene length as the biased-sampling term.
Enrichment is performed at multiple levels:

\begin{enumerate}
\def\labelenumi{\arabic{enumi}.}
\tightlist
\item
  Genes Mapped To All Windows

  \begin{enumerate}
  \def\labelenumii{\arabic{enumii}.}
  \tightlist
  \item
    Genes mapped in \emph{either comparison}
  \item
    Genes mapped in \emph{comparison 1 but not the second comparison}
  \item
    Genes mapped in \emph{comparison 2 but not the first comparison}
  \item
    Genes mapped in \emph{both comparisons}
  \end{enumerate}
\item
  Genes Mapped to Differential Signal Windows

  \begin{itemize}
  \tightlist
  \item
    All pair-wise combinations of Up/Down/Unchanged/Undetected are tested across both factors. If no enrichment is found, results are not presented.
  \end{itemize}
\end{enumerate}

Where enrichment for pathways are found, network plots are again generated

\hypertarget{integration-with-rna-seq-results}{%
\subsection*{Integration With RNA-Seq Results}\label{integration-with-rna-seq-results}}

The set of genes mapped to each pair-wise combination of Up/Down/Unchanged/Undetected are then compared to the external DE results using GSEA incorporating direction of change, and overall significance.
Differential expression is displayed as a multi-panel volcano plot, separating genes by detected signal patterns.
Along with barcode plots for the 9 most highly-ranked groups, genes in the Leading Edge for \emph{all} significant combined groups are also provided in the results table, along with network plots.
Significantly enriched pathways are again determined by combining p-values from GSEA performed on RNA-Seq and enrichment testing performed sites returning signal, using Wilkinson's method.

\hypertarget{key-outputs-2}{%
\subsection{Key Outputs}\label{key-outputs-2}}

Key output files produced by this step of the workflow include

\begin{itemize}
\tightlist
\item
  Pairwise differential signal patterns across both comparisons (\texttt{output/pairwise\_comparisons/\textless{}target1\textgreater{}\_\textless{}target2\textgreater{}/\textless{}target1\textgreater{}\_\textless{}comparison1\textgreater{}-\textless{}target2\textgreater{}\_\textless{}comparison2\textgreater{}-pairwise\_comparison.csv.gz})
\item
  The above file as an RDS object for easier parsing into an \texttt{R} environment.
\end{itemize}

If RNA-Seq data is provided, the file \texttt{output/pairwise\_comparisons/\textless{}target1\textgreater{}\_\textless{}target2\textgreater{}/\textless{}target1\textgreater{}\_\textless{}comparison1\textgreater{}-\textless{}target2\textgreater{}\_\textless{}comparison2\textgreater{}-de\_genes.csv} will contain mappings between DE genes and combined ChIP target differential signal patterns.
All lists of enriched pathways will also be exported as csv files.

\hypertarget{quick-start}{%
\chapter{Quick Start Guide}\label{quick-start}}

\hypertarget{snakemake}{%
\section{Install Snakemake}\label{snakemake}}

You will need a \texttt{snakemake} installation to begin.
Please see \href{https://snakemake.readthedocs.io/en/stable/getting_started/installation.html}{here} for help setting this up.
If you are running the pipeline on an HPC and are unsure, please consult with your HPC support team about setting up a \texttt{snakemake} profile on your specific cluster.

\hypertarget{quick-directories}{%
\section{Create the Directory Structure}\label{quick-directories}}

\begin{enumerate}
\def\labelenumi{\arabic{enumi}.}
\tightlist
\item
  Create a new \texttt{github} repository on your account by going to \href{https://github.com/smped/GRAVI/generate}{the github template repository}
\item
  Download your new repository to your local server or HPC using \texttt{git\ clone\ \textless{}myrepository\textgreater{}}
\item
  Place your bam files in the subdirectory \texttt{data/bam} or \texttt{data/aligned} as described in section \ref{alignments}
\item
  Edit \texttt{samples.tsv} in the \texttt{config} directory as described in section \ref{sample-descriptions}
\item
  Ensure you have the blacklist as a bed file and annotations as gtf
\item
  Modify any parameters in \texttt{config/config.yml}
\end{enumerate}

\hypertarget{quick-run}{%
\section{Run the Pipeline}\label{quick-run}}

\hypertarget{run-on-a-local-server}{%
\subsection{Run On A Local Server}\label{run-on-a-local-server}}

To run using 16 cores without any queuing system (e.g.~on a local machine), enter the following

\begin{verbatim}
snakemake -p --use-conda --notemp --keep-going --rerun-triggers mtime --cores 16
\end{verbatim}

\hypertarget{run-on-an-hpc}{%
\subsection{Run On An HPC}\label{run-on-an-hpc}}

Please consult with your local support team for their advice running a \texttt{snakemake} workflow.
In essence, the above command will need to be provided to your queuing system through the preferred strategy.
The \texttt{snakemake} profile required will generally be stable across all workflows but may require expertise from the technical support team.

\hypertarget{tips-and-tricks}{%
\section{Tips And Tricks}\label{tips-and-tricks}}

\hypertarget{removing-large-files}{%
\subsection{Removing Large files}\label{removing-large-files}}

Some large files, such as R Environments and BedGraph files are marked as \texttt{temp} files internally and these can be removed after completion of the workflow using

\begin{verbatim}
snakemake --delete-temp-output --cores 1
\end{verbatim}

\hypertarget{shared-conda-environments}{%
\subsection{Shared Conda Environments}\label{shared-conda-environments}}

\texttt{conda} environments can easily become bloated and if running multiple GRAVI analyses it may be simpler to host a common set of \texttt{conda} environments to avoid their constant recreation.
This can be performed by adding the argument \texttt{-\/-conda-prefix\ \textquotesingle{}/path/to/my/envs/\textquotesingle{}} in the call to \texttt{snakemake}

Conda environments can also be built prior to running the workflow using the standalone command

\begin{verbatim}
snakemake \
    --use-conda \
    --conda-prefix '/path/to/my/envs/' \
    --conda-create-envs-only \
    --cores 1
\end{verbatim}

\hypertarget{running-restricted-sections-of-the-worklow}{%
\subsection{Running Restricted Sections of the Worklow}\label{running-restricted-sections-of-the-worklow}}

Snakemake has the capacity to run a workflow up until a certain point and this can be easily done using the argument \texttt{-\/-until} and specifying the stage you wish to terminate the workflow at. For example, the argument \texttt{-\/-until\ compile\_macs2\_summary\_html} would only run the workflow until the macs2 summaries are compiled, which may be preferable for checking QC before proceeding to differential expression and pairwise comparisons.

\hypertarget{input-files}{%
\chapter{Files and Directories}\label{input-files}}

\hypertarget{directories}{%
\section{Directory Structure}\label{directories}}

The GRAVI workflow requires a set directory structure.
If using the template repository, as advised, this will be mostly taken care of.
The required directory structure is

\begin{verbatim}
project_home/
 ├── analysis
 ├── config
 ├── data
 ├── docs
 ├── output
 └── workflow
\end{verbatim}

\begin{itemize}
\tightlist
\item
  Rmarkdown scripts will be added to and executed from the \texttt{analysis} directory
\item
  Key configuration files are provided in the \texttt{config} directory
\item
  \emph{Your data} should be placed in the \texttt{data} directory as described below
\item
  The \texttt{html} output summarising all results will be produced in the \texttt{docs} directory
\item
  Additional output files will be placed in \texttt{output} with all figures written within the respective folder for each html page, or with \texttt{docs/assets}.
\item
  The workflow itself is run by all code supplied in the \texttt{workflow} directory
\item
  The complete \texttt{R\ Environment} for each compiled RMarkdown document is also saved in \texttt{output/envs}, and given their larger sizes, these can be deleted using \texttt{snakemake\ -\/-delete-temp-output\ -\/-cores\ 1} to conserve storage space.
\end{itemize}

\hypertarget{alignments}{%
\section{Alignments}\label{alignments}}

The GRAVI workflow currently takes \texttt{bam} files as the primary input.
Multiple workflows exist for quality control, adapter removal and de-duplication and it is assumed that supplied reads will have been pre-processed with the above steps, then aligned to the genome of interest.

Files should be placed in the \texttt{data/bam} directory as set in \texttt{config.yml}, although this can be changed if desired.

\begin{verbatim}
project_home/
 └── data
      └── bam 
           ├── target1_control_rep1.bam
           ├── target1_control_rep2.bam
           ├── target1_control_rep3.bam
           ├── target1_treat_rep1.bam
           ├── target1_treat_rep2.bam
           ├── target1_treat_rep3.bam
           ├── target2_control_rep1.bam
           ├── target2_control_rep2.bam
           ├── target2_control_rep3.bam
           ├── target2_treat_rep1.bam
           ├── target2_treat_rep2.bam
           ├── target2_treat_rep3.bam
           └── input1.bam 
\end{verbatim}

\hypertarget{sample-descriptions}{%
\section{Sample Descriptions}\label{sample-descriptions}}

The file \texttt{samples.tsv} defines the set of files which the workflow will be applied to.
Any files placed in the \texttt{data/aligned} directory, but not specified in this file will be ignored.
The desired layout should be a \emph{tab-delimited file} (i.e.~tsv).
These can be generated using Excel, Notepad++, R, Visual Studio, or any other software you are comfortable with.
A brief example would follow the layout

\begin{longtable}[]{@{}
  >{\raggedright\arraybackslash}p{(\columnwidth - 8\tabcolsep) * \real{0.3194}}
  >{\raggedright\arraybackslash}p{(\columnwidth - 8\tabcolsep) * \real{0.1389}}
  >{\raggedright\arraybackslash}p{(\columnwidth - 8\tabcolsep) * \real{0.1389}}
  >{\raggedright\arraybackslash}p{(\columnwidth - 8\tabcolsep) * \real{0.1667}}
  >{\raggedright\arraybackslash}p{(\columnwidth - 8\tabcolsep) * \real{0.1250}}@{}}
\toprule\noalign{}
\begin{minipage}[b]{\linewidth}\raggedright
sample
\end{minipage} & \begin{minipage}[b]{\linewidth}\raggedright
target
\end{minipage} & \begin{minipage}[b]{\linewidth}\raggedright
treat
\end{minipage} & \begin{minipage}[b]{\linewidth}\raggedright
replicate
\end{minipage} & \begin{minipage}[b]{\linewidth}\raggedright
input
\end{minipage} \\
\midrule\noalign{}
\endhead
\bottomrule\noalign{}
\endlastfoot
target1\_control\_rep1 & Target1 & Control & 1 & input1 \\
target1\_control\_rep2 & Target1 & Control & 2 & input1 \\
target1\_control\_rep3 & Target1 & Control & 3 & input1 \\
target1\_treat\_rep1 & Target1 & Treat1 & 1 & input1 \\
target1\_treat\_rep2 & Target1 & Treat1 & 2 & input1 \\
target1\_treat\_rep3 & Target1 & Treat1 & 3 & input1 \\
\end{longtable}

\hypertarget{required-columns}{%
\subsection{Required columns}\label{required-columns}}

This file must contain all four of the columns \texttt{sample}, \texttt{target}, \texttt{treat}, \texttt{input}, in any order.
If supplied, optional columns such as \texttt{replicate}, \texttt{passage} etc can be referenced in the workflow.
As well as defining all required steps for the workflow, labels for plots will be generated from combinations of these columns.

\begin{itemize}
\tightlist
\item
  \texttt{sample}: This \textbf{must} be identical to the filename, but without the \texttt{.bam} extension.
\item
  \texttt{treat}: This is used to define all comparisons
\item
  \texttt{input}: All files must correspond to a file in \texttt{data/bam} but without the \texttt{.bam} suffix. Each sample can have a separate input sample, or input samples can be shared across all or some of the samples.
\end{itemize}

\hypertarget{optional-columns}{%
\subsection{Optional Columns}\label{optional-columns}}

Any additional columns can be used to denote batches, or passages if running a nested/paired model.
These column names will be automatically detected at the appropriate steps of the workflow and incorporated into figures and tables.
Common column names may be \texttt{replicate} or \texttt{passage} (for cell lines)

\hypertarget{additional-files}{%
\section{Additional Files}\label{additional-files}}

Additional, optional files can also be supplied and is it customary to place these in \texttt{data/external} with paths (relative to \texttt{project\_home}) added to \texttt{config.yml}.
Full details are demonstrated in Section \ref{config-yml}
Names can be any informative name chosen by the user.

\begin{verbatim}
project_home/
 └── data
      ├── bam 
      └── external
           ├── gencode_annotation.gtf.gz
           ├── blacklist.bed.gz
           ├── rnaseq_topTable.tsv
           ├── external_features.gtf.gz
           ├── hic_interactions.bedpe
           ├── additional_coverage_control.bw
           └── additional_coverage_treat.bw
\end{verbatim}

\hypertarget{rna-seq}{%
\subsection{RNA-Seq}\label{rna-seq}}

Files provided with differential expression analysis results from a relevant RNA-Seq experiment should follow the layout as produced by \texttt{topTable()} from the \texttt{limma} package\citep{limma}.
Gene IDs should match those in the Gencode GTF (Ensembl IDs) and should be contained in a column called \texttt{gene\_id}.
Additional expected columns will be \texttt{logFC} and \texttt{FDR} or similar names which could be reasonably found by \texttt{regex} matching within the workflow.

\hypertarget{external-features}{%
\subsection{External Features}\label{external-features}}

These must be provided as a GTF which can be prepared by any method.
The feature types should be defined in a \textbf{field named \texttt{feature}.}
Non-overlapping features are optimal but not essential, and this is left to the users discretion.
For example, if providing features such as enhancers and super-enhancers\citep{rose}, it may be more sensible to provide these as mutually exclusive groups.

\hypertarget{hic-interactions}{%
\subsection{HiC Interactions}\label{hic-interactions}}

Significant interactions can be sourced using any methodology, however these must be provided in \texttt{bedpe} format.

\hypertarget{external-coverage}{%
\subsection{External Coverage}\label{external-coverage}}

Additional coverage files should be provided in \texttt{bigwig} format.

\hypertarget{editing-yaml}{%
\chapter{YAML Configuration Files}\label{editing-yaml}}

All YAML files which ruin the workflow are located in the \texttt{project\_home/config} directory.
The standard YAML structure is used in all files with the primary objective being passing workflow parameters to the various steps of the workflow.
There are four files which control various aspects: \texttt{config.yml}, \texttt{colours.yml}, \texttt{params.yml} and \texttt{rmarkdown.yml}

\hypertarget{config-yml}{%
\section{\texorpdfstring{The Main Configuration: \texttt{config.yml}}{The Main Configuration: config.yml}}\label{config-yml}}

This file sets many of the primary parameters and is the file which \emph{will need editing for any new dataset.}
Many settings should remain unchanged as changing default locations of files may lead to unexpected instability in the workflow, whilst other setting \textbf{should} be changed, such as those which determine which comparisons to perform.
This is the only file parsed directly by \texttt{snakemake} and subsequent rules, whilst all others are used to pass parameters to \texttt{R} environments.

An example layout of \texttt{config.yml} might be:

\begin{verbatim}
samples:
  file: "config/samples.tsv"

paths:
  bam: "data/bam"

genome:
  build: "GRCh37"

external:
  blacklist: "data/external/blacklist.bed.gz"
  gtf: "data/external/gencode.v33lift37.annotation.gtf.gz"
  rnaseq: "data/external/some_results.tsv"
  features: "data/external/custom_features.gtf.gz"
  hic: "data/external/hic_interactions.bedpe"
  coverage:
    H3K27Ac:
      control: "data/external/H3K27Ac_control.bw"
      treat: "data/external/H3K27Ac_treat.bw"

comparisons:
  method: 'sq-lt'
  fc: 1.2
  fdr: 0.05
  paired: false
  filter_q: 0.6
  contrasts:
    - ["control", "treat"]
  ihw: "regions"

peaks:
  macs2:
    gsize: "hs"
    fdr: 0.05
    keep_duplicates: "all"
  qc:
    outlier_threshold: 10
    allow_zero: true
    min_prop_reps: 0.5
\end{verbatim}

\hypertarget{settings-which-dont-need-to-be-modified}{%
\subsection*{Settings Which Don't Need To Be Modified}\label{settings-which-dont-need-to-be-modified}}

In general, the paths to key files don't need to be changed and default configurations are well tested.
Whilst varying these has been intermittently attempted successfully, unexpected instability may occur and as such, is discouraged.

\texttt{samples:} (Default: ``config/samples.tsv'')

By default, the file which contains all sample-level information is defined as \texttt{config/samples.tsv}.
This can be changed to reflect any requirements after inspecting QC reports.

\texttt{paths:}

\begin{itemize}
\tightlist
\item
  \texttt{bam:} (Default: ``data/bam'') Alignments should be placed in \texttt{data/bam} as advised in section \ref{alignments}, although this can be changed to any other path as desired.
\end{itemize}

\hypertarget{settings-which-should-be-modified}{%
\subsection*{Settings Which Should Be Modified}\label{settings-which-should-be-modified}}

\texttt{genome:}

Specify the genome build used for alignments and for gene annotations.

\texttt{external:}

Provide paths to all additional data files here.
Only files provided will be included in the workflow.

\begin{itemize}
\tightlist
\item
  \texttt{blacklist} Should be obtained from \href{https://github.com/Boyle-Lab/Blacklist/tree/master/lists}{here}
\item
  \texttt{gtf}: Should ideally be obtained from \href{https://www.gencodegenes.org/}{Gencode}
\item
  (Optional) \texttt{rnaseq} should be the results as output by \texttt{limma::topTable()} or similar. Gene IDs should match those provided in the Gencode GTF (e.g.~Ensembl IDs) and these should be in a column names \texttt{gene\_id}. Columns such as \texttt{logFC} and \texttt{FDR} will be searched for during the workflow using regular expressions to find the best match. Can \textbf{only} be in \texttt{csv} or \texttt{tsv} format. Excel-specific (\texttt{xls}, \texttt{xlsx}) formats are not supported.
\item
  (Optional) \texttt{features} can be determined by any method, with common choices being relevant histone marks, or promoters, enhancers and super-enhancers determined by H3K27ac marks. Features should be non-overlapping with the field \texttt{feature} defining the different feature types. Must be provided in GTF format.
\item
  (Optional) \texttt{hic} HiC interactions must be provided as a \texttt{bedpe} file
\item
  (Optional) \texttt{coverage} Tracks provided in this argument will be added to all genomic plots showing binding peaks or differential binding. Multiple files provided within each YAML list element will be overlaid as a single track. There is no upper limit to the number of tracks, however more tracks generally detract from an informative figure.
\end{itemize}

\texttt{comparisons:}

These settings determine how the differential binding analysis is performed.

\begin{itemize}
\tightlist
\item
  \texttt{method} (Default: ``sq-lt'') Defines the strategy used for differential binding. The currently implemented options are ``sq-lt'' and ``ls-ql'' as described previously
\item
  \texttt{fc} (Default: 1.2) The setting of 1.2 indicates a 20\% change in binding as the threshold below which we are not interested, or below which we consider binding changes to be inconsequential. This parameters is passed to \texttt{limma::treat()} \citep{treat} in all differential binding analyses.
\item
  \texttt{fdr} (Default: 0.05) Windows with significance below this threshold are considered to provide supporting evidence of differential binding.
\item
  \texttt{paired} (Default: \texttt{false}) If set to \texttt{true} the values in the optional column (e.g.~replicate, passage etc.) are used to perform a paired analysis as described in the \texttt{limma} manual
\item
  \texttt{filter\_q} (Default: 0.6) Passed to \texttt{extraChIPs::dualFilter()}. When filtering (i.e.~discarding) genomic sliding windows which are unlikely to contain true binding signal, determine thresholds which will retain this proportion of windows which overlap a peak identified by \texttt{macs2\ callpeak}.
\item
  \texttt{contrasts}: Define all contrasts for differential binding. Any ChIP target containing both treatment groups will be included for differential binding. Values must match those in the \texttt{treat} column of \texttt{samples.tsv}. Each differential binding analysis will be performed using the limma-trend method in the context of A vs B, such that complex models are not supported. Use new YAML list elements to define additional contrasts
\item
  \texttt{ihw} (Default: ``regions'') Options used to stratify p-values for Independent Hypothesis Weighting\citep{ihw} of differential binding results. Can take values \texttt{"regions"}, \texttt{"features"}, \texttt{"targets"} or \texttt{"none"}

  \begin{itemize}
  \tightlist
  \item
    \texttt{"regions"} P-values will be stratified by annotated genomic regions as determined in the initial steps of the workflow
  \item
    \texttt{"features"} P-values will be stratified by provided external features
  \item
    \texttt{"targets"} P-values will be stratified by the presence of a union peak using all other ChIP targets
  \item
    \texttt{"none"} No Independent Hypothesis Weighting will be performed on the results from differential binding
  \end{itemize}
\end{itemize}

\texttt{peaks:}

\texttt{macs2} settings are passed to \texttt{macs2\ callpeak}. Only the arguments \texttt{gsize}, \texttt{fdr} and \texttt{keep\_duplicates} are accepted.
Please see the \href{https://macs3-project.github.io/MACS/}{\texttt{macs2} manual} for more detailed explanations.

\texttt{qc} parameters are used for determining if samples are of a high enough quality, and how to determine union/treatment peaks for each target and treatment group.

\begin{itemize}
\tightlist
\item
  \texttt{outlier\_threshold} (Default: 10) As previously described, samples with peak numbers beyond this value relative to the median peak numbers in each treatment group are marked for removal. To ignore this parameter, set this to \texttt{.inf} which effectively sets the threshold to \(\infty\)
\item
  \texttt{allow\_zero} (Default: \texttt{true}) Allow samples with no identified peaks. If a ChIP target is expected to be cytoplasmic in one condition, this can allow samples with no peaks to be retained.
\item
  \texttt{min\_prop\_reps} (Default: 0.5) When forming \textbf{treatment-specific} peaks within each \emph{target and treatment group}, peaks must be represented in at least this proportion of samples. This defaults to 0.5 which would equate to 2 of 3 samples, 2 of 4 samples etc. This value may need to be altered pending the results of a complete run after Quality Assessment has performed.
\end{itemize}

\hypertarget{colour-schemes-colours.yml-colours-yml}{%
\section{\texorpdfstring{Colour Schemes: \texttt{colours.yml} {[}\#colours-yml{]}}{Colour Schemes: colours.yml {[}\#colours-yml{]}}}\label{colour-schemes-colours.yml-colours-yml}}

Defines all plotting colour schemes for consistency throughout the entire workflow.
Colours can be any standard colours able to be interpreted by \texttt{R}, such as \texttt{\textquotesingle{}blue\textquotesingle{}} or \texttt{\textquotesingle{}\#0000FF\textquotesingle{}}.
Recommended YAML list elements are \texttt{qc}, \texttt{regions}, \texttt{direction} and \texttt{treat}.
A gradient through al colours provided in the \texttt{heatmaps} element will be generated during generation of heatmaps.
Any unspecified colours will be automatically assigned and will propagate through the workflow.
As is standard across most programming languages, names are case-sensitive.
An example file is given below:

\begin{verbatim}
qc:
  pass: "#0571B0"
  fail: "#CA0020"
regions:
  promoter: '#FF3300'
  upstream_promoter: '#E1EE05'
  exon: '#7EDD57'
  intron: '#006600'
  proximal_intergenic: '#000066'
  distal_intergenic: '#551A8B'
treat:
  Input: "#33333380"
  E2: "#3333CC"
  E2DHT: "#C52240"
features:
  enhancer: "#FFFF00"
  no_feature: "#E5E5E5"
direction:
  up: "#CA0020"
  down: "#0571B0"
  unchanged: "#7F7F7F"
  undetected: "#E5E5E5"
heatmaps: ['white', 'red']
\end{verbatim}

\hypertarget{params-yml}{%
\section{\texorpdfstring{Additional Parameters: \texttt{params.yml}}{Additional Parameters: params.yml}}\label{params-yml}}

Default settings for defining initial annotations, mapping of peaks to genes and enrichment testing.
In general, these will not need to be changed, but can be if required.
In particular, the value \textbf{enh2gene} can be set to zero if using data such as H3K27ac HiChIP to determine long-range interactions.

For enrichment testing, the preferred p-value adjustment method can be specified using any value that match \texttt{p.adjust.methods} with an appropriate threshold also set in this section.
Default settings are relatively inclusive and more stringent setting may be preferred by some users.

Parameters for \texttt{msigdb}\citep{msigdb} are passed to \texttt{msigdbr}\citep{msigdbr} and fields should match this layout.
Any categories passed to \texttt{gs\_cat} will lead to all subcategories being used from that category.
Specific sub-categories of larger databases can be passed using \texttt{gs\_subcat} element.

Values which determine the minimum and maximum size of any network plots are also defined here.
Node-pair distances above the given distance will be removed from the plot and these edges excluded.

\begin{verbatim}
gene_regions:
  promoters:
    upstream: 1500
    downstream: 500
  upstream: 5000
  intergenic: 10000

## The values used when mapping peaks to genes.
## Passed to `extraChIPs::mapByFeature()`
## If including H3K27ac HiChIP for long range-interactions, it is advised to
## set `enh2gene` as zero, given that long range interactions in this case
## will more accurately map long-range enhancer interactions than using all
## genes within a given distance
mapping:
  gr2gene: 100000
  prom2gene: 0
  enh2gene: 100000
  gi2gene: 0

enrichment:
  adj: "fdr"
  alpha: 0.05
  ## Only gene-sets between these two values will be retained before testing
  min_size: 5
  max_size: 1000
  ## Only gene-sets above this size will be shown in the results
  min_sig: 3
  ## The categories to use from MSigDB. These are passed to msigdbr in the
  ## columns of the same name in an 'OR' approach
  species: "Homo sapiens"
  msigdb:
    gs_cat: "H"
    gs_subcat:
      - "CP:KEGG"
      - "CP:REACTOME"
      - "CP:WIKIPATHWAYS"
      - "TFT:GTRD"

## Used for network plots
networks:
  min_size: 4
  max_size: 80
  max_distance: 0.9
  layout: 'fr'
\end{verbatim}

\hypertarget{rmarkdown-yml}{%
\section{\texorpdfstring{HTML Settings: \texttt{rmarkdown.yml}}{HTML Settings: rmarkdown.yml}}\label{rmarkdown-yml}}

The main workflow will produce a compiled set of HTML pages using \texttt{rmarkdown::render\_site()}\citep{R-rmarkdown}.
The two available fields to supply here are:

\begin{enumerate}
\def\labelenumi{\arabic{enumi}.}
\tightlist
\item
  \texttt{knitr\_opts} which are passed to \texttt{knitr::opts\_chunk\$set()}\citep{R-knitr} at the beginning of every compiled Rmarkdown document, and
\item
  \texttt{rmarkdown\_site} which determines the layout and style of the final HTML report.
  All \emph{left} elements of the \texttt{navbar} are determined automatically during the workflow and will be ignored if supplied here, whilst all other parameters are passed via the file \texttt{\_site.yaml} which will be generated during the workflow.
\end{enumerate}

Figure height and width parameters must be passed to \texttt{knitr::opts\_set()} in inches, and the workflow defaults to \texttt{svg} output for most figures.
Some figures which can become excessively large under this setting, such as those with many thousands of overlapping points, are set to always produce \texttt{png} output.
Multiple output formats can also be set using values such as \texttt{{[}\textquotesingle{}png\textquotesingle{},\ \textquotesingle{}pdf\textquotesingle{}{]}} if preferred.

\begin{verbatim}
knitr_opts:
  echo: TRUE
  message: FALSE
  warning: FALSE
  dev: ["png", "pdf"]
  fig.align: "center"
  fig.width: 10
  fig.height: 8

rmarkdown_site:
  name: "GRAVI: Gene Regulatory Analysis"
  output_dir: "../docs"
  navbar:
    title: "GRAVI"
    right:
      - icon: fa-github
        href: "https://github.com/smped/GRAVI"
  output:
    html_document:
      toc: yes
      toc_float: yes
      code_folding: hide
      self_contained: false
      theme: sandstone
      highlight: textmate
      css: gravi.css
      includes:
        after_body: footer.html
\end{verbatim}

  \bibliography{book.bib,packages.bib}

\end{document}
